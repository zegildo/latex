\documentclass[a4paper, 10pt]{report}
\usepackage{bsi_angicos}

\begin{document}

\ata{ATA DA PRIMEIRA REUNIÃO ORDINÁRIA DO ANO DE DOIS MIL E DEZENOVE DO COLEGIADO DO CURSO DE BACHARELADO EM SISTEMAS DE INFORMAÇÃO DA UNIVERSIDADE FEDERAL RURAL DO SEMI-ÁRIDO, CENTRO MULTIDISCIPLINAR DE ANGICOS.}{

Aos vinte e quatro dias do mês de janeiro de dois mil e dezenove, às dezesseis horas e trinta minutos, na sala de número dois do bloco de professores de número um, reuniu-se o Colegiado do Curso de Bacharelado em Sistemas de Informação (BSI) da Universidade Federal Rural do Semi-Árido (UFERSA), do Centro Multidisciplinar de Angicos, sob a presidência do professor \textbf{José Gildo de Araújo Júnior} para deliberar sobre a pauta da primeira reunião ordinária de dois mil e dezenove. Estiveram presentes: \textbf{Francisco de Assis Pereira Vasconcelos de Arruda} (vice-presidente do colegiado), \textbf{Maria das Neves Pereira}, \textbf{Joêmia Leilane Gomes de Medeiros} e \textbf{Adriana Mara Guimarães de Farias}, representantes do corpo docente. Os seguintes membros estiveram ausentes: \textbf{Patrício de Alencar Silva} e \textbf{Bruno Bandeira Duarte}. \textbf{PAUTA: Primeiro ponto:} Aprovação do horário dois mil e dezenove, ponto, um. \textbf{Segundo ponto:} Providências em relação ao \textit{LaComp}. \textbf{Terceiro ponto:} Exame Nacional de Desempenho de Estudantes (ENADE) dois mil e dezenove. \textbf{Quarto ponto:} Encaminhamentos. O presidente do conselho, o professor \textbf{José Gildo de Araújo Júnior}, colocou a pauta em votação, sendo aprovada por unanimidade. \textbf{PRIMEIRO PONTO.} A representante docente, a professora \textbf{Maria das Neves Pereira} questionou sobre a necessidade de modificação do horário. O presidente do conselho, o professor \textbf{José Gildo de Araújo Júnior} expôs que a necessidade em se ajustar o horário é decorrente da dinâmica da universidade, onde alguns professores são removidos para outros \textit{campi}, outros professores saem em consequência do Plano de Qualificação Docente (PQD), alguns cursos sofrem modificações em seu Projeto Pedagógico de Curso (PPC), havendo, assim, a necessidade de restruturação do horário. A representante docente, a professora \textbf{Adriana Mara Guimarães de Farias} sugeriu que a disciplina Paradigmas de Programação (PP), do período dois mil e dezenove, ponto, um, fosse alterada para a quarta-feira no turno da manhã ao invés de manter-se na quarta-feira no turno da tarde. O horário dois mil e dezenove, ponto, um, foi votado e aprovado por unanimidade. \textbf{SEGUNDO PONTO.} O vice-presidente do colegiado, o professor \textbf{Francisco de Assis Pereira Vasconcelos de Arruda} sugeriu um protocolo de funcionamento do Laboratório de Computação (\textit{LaComp}), onde a chave do laboratório deveria estar na portaria da instituição, de modo a delegar ao porteiro a identificação de alunos-curadores, contando com registro de entrada e saída. O presidente do colegiado, o professor \textbf{José Gildo de Araújo Júnior} colocou a proposta do conselheiro em votação, sendo aprovada por unanimidade. \textbf{TERCEIRO PONTO.} A representante docente, a professora \textbf{Maria das Neves Pereira} sugeriu a criação de um cronograma de atividades para o ENADE dois mil e dezenove. O vice-presidente do colegiado, o professor \textbf{Francisco de Assis Pereira Vasconcelos de Arruda} sugeriu publicizar recomendações aos professores do curso de modo a abordarem em seus planos de aula questões e atividades relacionadas ao ENADE dois mil e dezenove. Ambas as propostas foram postas em votação e aprovadas por unanimidade. \textbf{QUARTO PONTO.} A representante docente, a professora \textbf{Joêmia Leilane Gomes de Medeiros} sugeriu a criação de critérios para decidir sobre situações de impasse quando dois ou mais professores manifestam interesse por lecionar a mesma disciplina. O presidente do colegiado, o professor \textbf{José Gildo de Araújo Júnior} comentou que a decisão de alocação de professores às disciplinas compete exclusivamente à Chefia de Departamento e que o colegiado de curso poderia, unicamente, encaminhar suas sugestões. O encaminhamento foi posto em votação e aprovado por unanimidade. Após considerações finais, o presidente do colegiado, o professor \textbf{José Gildo de Araújo Júnior}, agradeceu a presença de todos os membros presentes e deu por encerrada a reunião. E eu, \textbf{José Gildo de Araújo Júnior}, secretário \textit{ad hoc}, lavrei a presente Ata, que após lida e aprovada sem emendas, na reunião do dia XX de XX de dois mil e dezenove, segue assinada pelo presidente do colegiado de curso e pelos demais conselheiros presentes a esta reunião e por mim.

\textbf{Presidente:}

José Gildo de Araújo Júnior

\textbf{Vice-Presidente:}

Francisco de Assis Pereira Vasconcelos de Arruda

\textbf{Conselheiras:}

Maria das Neves Pereira

Joêmia Leilane Gomes de Medeiros

Adriana Mara Guimarães de Farias

}
\end{document}